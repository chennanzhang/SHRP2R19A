\documentclass{standalone}
\usepackage{tikz,ctex}
\setCJKmainfont[BoldFont={NotoSerifCJKsc-SemiBold.otf}]{NotoSerifCJKsc-Light.otf}
\usetikzlibrary{positioning,shapes.geometric,shapes.misc,shapes.symbols,shapes.arrows}

\tikzset{
  >=stealth,
  methodbase/.style ={
    rectangle,
    rounded corners=3pt,
    node distance=12mm,
    inner sep=5pt,
    minimum width=3cm,
    minimum height=1.0cm,
  },
  sys/.style={
    methodbase,
    draw=red!50!black,
    fill=red!50!black!20,
  },
  subsys/.style={
    methodbase,
    draw=brown!50!black,
    fill=brown!50!black!20,
  },
  comp/.style={
    methodbase,
    draw=blue!50!black,
    fill=blue!50!black!20,
  },
  ele/.style={
    methodbase,
    draw=green!50!black,
    fill=green!50!black!20,
  }
}
\begin{document}\footnotesize
  \begin{tikzpicture}
    \coordinate (A) at (0,-0.9);
    \node (root) [comp] {下部结构组件};
    \node (pier) at ([xshift=-2.5cm,yshift=-1.2cm]root.south)[subsys] {桥墩子系统};
    \coordinate (B) at ([xshift=-1cm]pier.south);
    \node (abutment) at ([xshift=2.5cm,yshift=-1.2cm]root.south)[subsys]{桥台子系统};
    \coordinate (C) at ([xshift=-1cm]abutment.south);
    \node (P) at ([xshift=1cm,yshift=-1cm]pier.south) [ele] {
      \parbox{8em}{盖梁、立柱\\ 承台/桩帽 \\ 钻孔、桩基础}
    };
    \node (AB) at ([xshift=1cm,yshift=-1.4cm]abutment.south) [ele] {
      \parbox{8em}{背墙、帽梁\\ 前墙、翼墙\\ 承台/桩帽 \\ 钻孔、桩基础 \\ 加筋土}
    };
    \draw(root)--(root|-A);
    \draw(pier)--(pier|-A)--(A-|abutment)--(abutment);
    \draw(B)--(B|-P)--(P);
    \draw(C)--(C|-AB)--(AB);
  \end{tikzpicture}
\end{document}