\documentclass{standalone}
\usepackage{tikz,ctex}
\setCJKmainfont[BoldFont={NotoSerifCJKsc-SemiBold.otf}]{NotoSerifCJKsc-Light.otf}
\usetikzlibrary{positioning,shapes.geometric,shapes.misc,shapes.symbols,shapes.arrows}
\tikzset{
  >=stealth,
  start/.style ={
    rounded rectangle, 
    node distance=12mm,
    minimum width =50pt,
    minimum height =15pt,
    inner sep=3pt,
    draw=cyan!80!black,
    fill=cyan!50!black!20
  },
  methodbase/.style ={
    rectangle,
    rounded corners=3pt,
    node distance=12mm,
    inner sep=5pt ,
    draw=green!50!black,
    fill=green!50!black!20,
  },
  method/.style={
    % minimum width=3.5cm,
    methodbase,
  },
  input/.style ={
    trapezium, 
    trapezium left angle=60, 
    trapezium right angle=120,
    draw=yellow!50!black,
    fill=yellow!50!black!20,
    minimum height =15pt,
    inner sep=3pt
  },
  end/.style ={
    rounded rectangle,
    node distance=12mm,
    minimum width =50pt,
    minimum height =15pt,
    inner sep=3pt ,
    draw=blue!80!black,
    fill=blue!50!black!20
  },
  theory/.style ={rectangle,node distance=30mm,
  minimum width =50pt ,minimum height =15pt ,inner sep=3pt ,
  draw=red!50!black ,fill=red!50!black!20},
  branch/.style ={diamond,node distance=20mm,aspect=2.2,
  %minimum width =50pt ,minimum height =15pt ,inner sep=3pt ,
  draw=red!50!black ,fill=red!50!black!20},
  root/.style ={
    circle,draw,
    minimum height=1cm,
    draw=orange!80!black,
    fill=orange!50!black!20
  }
}
\begin{document}\footnotesize
  \begin{tikzpicture}
    \node (start) [start] {使用寿命设计的一般步骤};
    \node (step1) [method, below= 3mm of start] {\parbox{4cm}{1. 确定工作和使用寿命要求。}};
    \node (step2) [method, below= 3mm of step1] {\parbox{4cm}{2. 确定满足 LRFD 规范设计规定的可行桥梁系统备选方案。}};
    \node (step3) [method, below= 3mm of step2] {\parbox{6cm}{3. 根据《指南》各章节中所述的使用寿命要求,评估所选桥梁系统的所有部件、构件和子系统。}};
    % \node (c) [root, right= 1cm of step3] {C};
    \node (step4a) [branch, below= 3mm of step3] {\parbox{1.6cm}{4a. 使用寿命是否适用?}};
    \node (step4b) [method, right= 5mm of step4a] {\parbox{2.3cm}{4b. 确定缓解策略并修改桥梁设计。}};
    \node (step4c) [branch, below= 3mm of step4a] {\parbox{1.8cm}{4c. 使用寿命需求均已考虑?}};
    % \node (gotoa) [root, below= 3mm of step4c] {A};
    \node (step5) [method, below= 3mm of step4c] {\parbox{4cm}{5. 开发满足 LRFD 规范和使用寿命要求的改进型桥梁系统。}};
    \node (step6) [method, right= 3mm of step5] {\parbox{4cm}{6. 确定所选桥梁备选方案是否满足要求以及与设计更改是否兼容。}};
    \node (step7a) [branch, below= 3mm of step6] {\parbox{1.8cm}{7a. 备选方案是否满足要求?}};
    \node (step7b) at (step7a-|step5)[method] {\parbox{4cm}{7b. 对桥梁部件、子系统或构件进行适当的修改以实现兼容性。}};
    \node (step8) [start,below= 3mm of step7a] {8. 开发所选桥梁方案的最终配置。};
    \node (step9a) [method, below= 3mm of step8] {\parbox{3.6cm}{9a. 预测各种部件、子系统和构件的使用寿命}};
    \node (step10) [method,below = 6mm of step9a] {\parbox{3.6cm}{10. 编制桥梁系统备选方案的所有要求并计算生命周期成本。}};
    \node (step9d) at (step5|-step10) [method] {\parbox{2.5cm}{9d. 确定维护计划。}};
    \node (step9b) at (step9a-|step5)[branch] {\parbox{1.8cm}{9b. 使用寿命—部分大于系统?}};
    \node (step9c) [method, left = 5mm of step9b] {\parbox{1.6cm}{9c. 确定修复或更换计划。}};
    \node (step11a) [branch, below= 3mm of step10] {\parbox{1.8cm}{11a. 所有备选方案均已考虑?}};
    \node (step11b) [method, right= 7mm of step11a] {\parbox{1.7cm}{11b. 考虑下一备选方案。}};
    \node (step12) [end, left= 5mm of step11a] {12. 比较所有最终备选方案的优缺点并选择最终方案。};
    \draw [->](start)--(step1);
    \draw [->](step1)--(step2);
    % \draw [->](c)--(step3);
    \draw [->](step2)--(step3);
    \draw [->](step3)--(step4a);
    \draw [->](step4b)--(step4b|-step4c)--(step4c);
    \draw [->](step4a)--(step4c) node [midway,right]{\scriptsize 否};
    \draw [->](step4a)--(step4b) node [midway,above]{\scriptsize 是};
    \draw [->](step4c)--(step5) node [midway,right]{\scriptsize 是};
    \coordinate (A) at ([xshift=-2cm]step4c.west);
    \coordinate (B) at ([xshift=4mm]step9b.east);
    \draw [->](step4c)--(A) node [at start,above left]{\scriptsize 否}--(A|-step3)--(step3);
    \draw [->](step5)--(step6);
    \draw [->](step6)--(step7a);
    \draw [->](step7a)--(step8) node [midway,right]{\scriptsize 是};
    \draw [->](step7a)--(step7b) node [midway,above]{\scriptsize 否};
    \draw [->](step7b)--(step7b|-step8)--(step8);
    \draw [->](step8)--(step9a);
    \draw [->](step9a)--(step9b);
    \draw [->](step9b)--(step9d) node [midway,right]{\scriptsize 是};
    \draw [->](step9b)--(step9c) node [midway,above]{\scriptsize 否};
    \draw [->](step9c)--(step9c|-step9d)--(step9d);
    \draw [->](step9d)--(step10);
    \draw [->](step10)--(step11a);
    \draw [->](step11a)--(step12) node [midway,above]{\scriptsize 是};
    \draw [->](step11a)--(step11b) node [midway,above]{\scriptsize 否};
    \draw [->](step11b)--(step11b|-step3)--(step3);
  \end{tikzpicture}
\end{document}