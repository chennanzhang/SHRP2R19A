\usepackage[acronym]{glossaries}
\renewcommand*{\glossaryname}{词汇表}
\renewcommand*{\acronymname}{缩略语}
\makeglossaries
\newglossaryentry{ActiveX}{
  name = ActiveX,
  type=main,
  description ={(孩子,你去问\Microsoft{})}
}
\newglossaryentry{Bookmark}{
  name = 书签,
  description = {放置在文档中的位置标记,让用户能够快速返回到该位置。}
}
\newglossaryentry{Breakpoint}{
  name = 断点,
  description = {放置在程序代码中的标记,指示编译器或解释器在运行时暂停执行并等待用户执行调试任务或继续执行。}
}
\newglossaryentry{Callback}{
  name = 回调,
  description = {对给定操作或事件的请求响应。例如,单击按钮的回调可能是“接受”,然后触发对特定函数或表达式的调用。}
}
\newglossaryentry{Collection}{
  name = 类集,
  description = {具有共同父对象和相关属性或方法的一组对象,这些属性或方法支持以逻辑方式将对象作为一个组进行处理。}
}
\newglossaryentry{COM}{
  name = COM,
  description = {组件对象模型。\Microsoft{} 的一项技术,它定义了软件组件和服务的分层组织,并为组件和服务提供了内在属性、方法和事件,以实现更高效的编程操作并促进组件化的功能重用。其他类型包括分布式 COM 或 DCOM 以及 \Windows{} 2000 和 XP 平台中包含的较新的 COM+。}
}
\newglossaryentry{Constant}{
  name = 常数,
  description = {具有静态赋值的变量或符号。}
}
\newglossaryentry{Consumer}{
  name = 消费者,
  description = {
    导入或使用另一个软件组件或服务的公开组件服务的任何软件组件或应用程序。导入的组件或服务的来源称为提供者}
}
\newglossaryentry{Control}{
  name = 控件,
  description = {一个 \AX{} DLL 组件。}
}
\newglossaryentry{Data Type}{
  name = 数据类型,
  description = {与特定值表示的数据形式有关的内在性质。 \AX{} 数据类型的示例包括整数 (\Str{Integer})、长整数(\Str{Long})、双精度浮点数(\Str{Double})、字符串(\Str{String}) 和数组(\Str{Array})。}
}
\newglossaryentry{DCL}{
  name = DCL,
  description = {对话框控制语言,基于 C 语言构造,用于在 \ALSP{} 和 \VLSP{} 环境中定义对话框形式。}
}
\newglossaryentry{Debug}{
  name = 调试,
  description = {隔离、诊断和纠正程序代码或编程逻辑中错误的过程。}
}
\newglossaryentry{Dictionary}{
  name = 目录,
  description = {一种集合类型,它通过为每个对象使用唯一标识符来提供对成员对象的直接访问。}
}
\newglossaryentry{DLL}{
  name = 动态链接库,
  description = {动态链接库,基于 \Windows{} 的 \AX{} 组件,通常公开函数、属性、方法和常量以供其他应用程序使用。它类似于一个打包的工具库,应用程序可以加载这些工具来执行专门的任务。}
}
\newglossaryentry{Element}{
  name = 元素,
  description = {数组或安全数组构造的单个成员。}
}
\newglossaryentry{Enumeration}{
  name = 枚举,
  description = {(这个需要官方定义)}
}
\newglossaryentry{Evaluate}{
  name = 求值,
  description = {执行 LISP 表达式或从 LISP 符号中提取关联值并返回结果的过程。}
}
\newglossaryentry{Event}{
  name = 事件,
  description = {程序中发生某些动作的时刻。可以是单击按钮或移动实体。事件通常提供可以使用反应器或回调检测和响应的编程通知。}
}
\newglossaryentry{Expression}{
  name = 表达式,
  description = {\ALSP{} 或 \VLSP{} 解释器环境上下文中的程序语句。}
}
\newglossaryentry{Focus}{
  name = 焦点,
  description = {DCL 对话框中给定项的状态或者由活动光标位置控制。如果编辑框的光标处于活动状态并且是可编辑的,则称其具有焦点。当光标移出给定项目时,称其已失去焦点。}
}
\newglossaryentry{Function}{
  name = 函数,
  description = {在软件开发的上下文中,这是一个表达式或一组表达式,用于处理某种类型的输入并返回结果。在 \VLSP{} 的上下文中,子例程和函数是同义词。在 Visual Basic 或 C/C++ 等其他语言的上下文中,子例程不返回结果,而函数返回结果。}
}
\newglossaryentry{Global}{
  name = 全局,
  description = {在同一名称空间中运行的,可由所有其他变量、符号或表达式公开以供读取或写入操作的任何变量、符号或表达式。其他表达式无法访问的符号或表达式被称为局部于其父函数或表达式。}
}
% \newglossaryentry{}{
%   name = ,
%   description = {}
% }
% \newglossaryentry{}{
%   name = ,
%   description = {}
% }
% \newglossaryentry{}{
%   name = ,
%   description = {}
% }
% \newglossaryentry{}{
%   name = ,
%   description = {}
% }
% \newglossaryentry{}{
%   name = ,
%   description = {}
% }
% \newglossaryentry{}{
%   name = ,
%   description = {}
% }
% \newglossaryentry{}{
%   name = ,
%   description = {}
% }
% \newglossaryentry{}{
%   name = ,
%   description = {}
% }
% \newglossaryentry{}{
%   name = ,
%   description = {}
% }
% \newglossaryentry{}{
%   name = ,
%   description = {}
% }
% \newglossaryentry{}{
%   name = ,
%   description = {}
% }
% \newglossaryentry{}{
%   name = ,
%   description = {}
% }
% \newglossaryentry{}{
%   name = ,
%   description = {}
% }
% \newglossaryentry{}{
%   name = ,
%   description = {}
% }
% \newglossaryentry{}{
%   name = ,
%   description = {}
% }
% \newglossaryentry{}{
%   name = ,
%   description = {}
% }
% \newglossaryentry{}{
%   name = ,
%   description = {}
% }
% \newglossaryentry{}{
%   name = ,
%   description = {}
% }
% \newglossaryentry{}{
%   name = ,
%   description = {}
% }
% \newglossaryentry{}{
%   name = ,
%   description = {}
% }
% \newglossaryentry{}{
%   name = ,
%   description = {}
% }
% \newglossaryentry{}{
%   name = ,
%   description = {}
% }
% \newglossaryentry{}{
%   name = ,
%   description = {}
% }
% \newglossaryentry{}{
%   name = ,
%   description = {}
% }
\newacronym{html}{HTML}{hypertext markup language}
\newacronym{html}{HTML}{hypertext markup language}
