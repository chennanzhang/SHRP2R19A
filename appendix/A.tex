% \chapter{Design Provisions for Self-stressing System for Bridge Application with Emphasis on Precast Panel Deck System}
\chapter{以预制面板系统为重点的桥梁应用自应力系统设计规定}
% Steel girder bridges often use continuity over the interior supports to reduce interior forces on the spans. In continuous structures with composite concrete decks, the location of maximum negative bending moment is over the interior supports. This moment produces tensile stresses in the concrete deck and compressive stress in the bottom flanges of the girders. The tensile stress in the deck leads to cracking, which allows intrusion of moisture and road salt, causing corrosion of the reinforcement, and supporting girders. Continued maintenance is required to forestall the deterioration; however, replacement of the deck is eventually required.
钢梁桥通常使用内部支撑的连续性来减少跨度上的内力。 在具有组合混凝土桥面板的连续结构中,最大负弯矩的位置在内部支撑上方。 这个力矩在混凝土面板中产生拉应力,在大梁的底部翼缘板中产生压应力。桥面板中的拉应力会导致混凝土开裂,从而使水分和道路盐分侵入,导致钢筋和支撑梁腐蚀。 虽然采取持续维护可以防止结构\gls*{deterioration},但最终还是需要更换桥面板。

% To help alleviate this problem, a self-stressing system was developed as part of the SHRP 2 R19A project. Additional details of the development can be found in the forthcoming final report. The method induces a compressive force in the deck accomplished by raising the interior supports above their final elevation while the deck is cast (cast-in-place construction) or placed (precast construction). Once the concrete has cured, the supports are lowered to their final elevation. Continuity of the steel member and the composite action with the deck produce a compressive stress in the concrete slab, which is balanced by tensile stresses in the bottom of the steel member. As a result, the cracking over interior support is reduced, increasing durability. Additionally, the need for girder splices may be eliminated, making the overall bridge design more efficient and less expensive when compared to conventional design.
为了帮助缓解这个问题,\gls{shrp}2 的 R19A 项目开发了一个自应力系统。有关开发的更多详细信息,请参阅即将发布的最终报告。 该方法通过在浇筑(现场浇筑施工)或安放(预制施工)桥面板时将内部支撑升高到其最终高度以上来在桥面板中产生压缩力。 一旦混凝土固化,支架就会降低到最终高度。钢构件的连续性和与桥面板的复合作用在混凝土板中产生压应力,该压应力由钢构件底部的拉应力平衡。 结果,减少了内部支撑的开裂,增加了耐用性。 此外,与传统设计相比,可以消除对大梁拼接的需要,从而使整体桥梁设计更加高效且成本更低。


% This appendix describes the construction procedure, design considerations, and implementation details for using the self-stressing method. A flow chart is provided to aid in implementation. Simplified formulas applicable to twospan bridges, which represent the most likely use of the method, are also included.
本附录描述了使用自应力法的施工程序、设计注意事项和实施细节。 提供流程图以帮助实施。 还包括适用于双跨桥梁的简化公式,代表最有可能使用该方法。

% \section{CONSTRUCTION PROCEDURE OVERVIEW}
\section{施工方法概述}
% This section provides a brief description of the major steps in the construction procedure, in order to establish a frame of reference and to introduce vocabulary used throughout the appendix. Table A.1 illustrates the major steps required for constructing a bridge using the self-stressing method. These steps will be used as points of reference in the remaining discussion.
本节简要描述构建过程中的主要步骤,以建立参考框架并介绍整个附录中使用的词汇。 \cref{tab:self-stressing-method} 说明了使用自应力法建造桥梁所需的主要步骤。 这些步骤将用作其余讨论的参考点。

\begin{table}
  \caption{Self-Stressing Method Major Steps}\label{tab:self-stressing-method}
  % \input{tables/filename}
\end{table}

The first stage consists of simply placing the girder onto the level supports. The resulting moments and deflections are those obtained from a continuous beam analysis.

During the second stage, the interior support is raised. The bare steel girder responds as a simply supported beam subjected to an upward directed point load at the location of the interior support. Note that the supports could be in the raised position prior to placing the girder. However, due to superposition, the analysis would be the same as described.


Next the concrete deck is cast, or precast panels are placed and grouted. The structure responds like a continuous bare steel beam, just as it would be for conventional construction.

During the fourth stage, the interior support is lowered to its final position. Just as in Stage 2, the response is that of a beam supported at the exterior supports only and subjected to a point load. However, the structure is now composite and the point load is directed downward. This action places the concrete deck over the supports into compression.

Over time, creep and shrinkage occur in the concrete deck. This may be accounted for in two stages. First, the creep and shrinkage are seen as an applied curvature on the structure. If the beam were simply supported by the exterior supports, this applied curvature would result in additional deflection without inducing additional load. However, due to the continuity, a restoring force is generated that prevents the displacement and results in additional stresses.

% \section{DESIGN CONSIDERATIONS}
\section{设计注意事项}
This section provides a discussion of the design issues specific to the use of the self-stressing method. Design of bridges using the self-stressing method should follow the provisions for I-Section and Box-Section flexural members contained in Section 6.10 and 6.11 respectively, of the AASHTO LRFD Bridge Design Specifications (LRFD Specifications), except as modified herein \cite{aashto2012l}.

\subsection{General}
The use of the self-stressing method is limited to straight I- and Box-Section steel girders and is applicable only to continuous multi-span structures with a composite deck. Practical limitations dictate that the method is most likely to be used in two-span structures. Simplified design aids are provided in Section A.5 for structures with two spans.

\subsection{Analysis}
Two options provided for the analysis of the structure are described in the following section. Note that the analysis methods should only be used when analyzing the construction steps associated with the self-stressing method and not the overall analysis procedures as covered in Chapter 4 of the LRFD Specifications.

\subsubsection{Simplified}
The simplified analysis method relies on first order techniques that disregard time effects in the concrete. These effects are accounted for using conservative correction factors presented in the implementation details portion of the provisions (Section A.3). The correction factors account for the effects of creep and shrinkage in the evaluation of stresses and deflections. As an alternative, advanced methods of analysis may be used that directly evaluate these effects.


\subsubsection{Advanced}

Advanced methods explicitly consider the effects of creep and shrinkage to evaluate the stresses and deflections.

Several examples are the effective modulus method (EMM), adjusted effective modulus method (AEMM), stepby-step method (SSM), and the rate of creep method (RCM).

When the creep and shrinkage strains are known, or otherwise assumed, then the LRFD Specifications, Section C4.6.6 can be used for calculating the resulting stresses and deformations.

\subsection{Forces}

The forces and stresses in all components that arise due to the self-stressing construction procedure should be considered in evaluating the load effects during design. For the purpose of design, the locked-in prestressing force shall be considered a dead load force applied to the composite long-term section (DC2).


The LRFD Specifications, Section 3.4.1 states that where prestressed components are used in conjunction with steel girders, the force effect should be considered as locked-in construction loads (EL). However, in this situation the prestressing forces are being developed by gravity effects rather than applied by prestressing devices. As such, the variability in the resulting stresses will be of the same magnitude as the variability of the dead load effects, which leads to the decision of considering the prestress stress as DC2 loading.


Note that the self-stressing procedure will generate tensile stresses in the bottom of the steel girders that will serve to offset some of the compressive dead and live load stresses. As such, the stresses due to the self-stressing procedure should be kept separate from other dead load stress sources and the minimum load factor for dead load should be used (0.9).

\subsection{Deflections}

The final deflected shape is necessary for determining the camber requirements of the girders and can be obtained by summing the deflections from the various construction stages.

\section{DESIGN PROCEDURE AND IMPLEMENTATION DETAILS}

This section provides a step-by-step procedure for designing a bridge incorporating the self-stressing method. All grout, and/or adhesives must be adequately cured prior to lowering the interior support. The creep and shrinkage properties of the materials must be compatible with the intended use and properties assumed during analysis.

\paragrah*{Step1. Determine Required Amount of Prestress}
The self-stressing method is a way to introduce compressive stresses in the concrete deck of a multi-span
continuous beam. The compressive stresses are generally located near the interior supports and therefore work to
counter the tensile stresses that arise in this vicinity due to gravity and live loading. The result is a reduction in
cracking and an accompanying increase in service life. The magnitude of the prestress that must be applied in order
to achieve the desired effects has been determined based on past experience with decks that have been prestressed
using traditional mechanical methods.


\paragrah*{step1}
\paragrah*{step1}
\paragrah*{step1}
\paragrah*{step1}
\paragrah*{step1}


% \section{DESIGN FLOWCHART}
\section{设计流程图}

% \section{DESIGN AIDS FOR TWO-SPAN BRIDGES}
\section{两跨桥梁的设计辅助工具}